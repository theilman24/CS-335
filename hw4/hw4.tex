\documentclass[11pt]{amsart}
%\geometry{landscape}                % Activate for for rotated page geometry
\usepackage[parfill]{parskip}    % Activate to begin paragraphs with an empty line rather than an indent
\usepackage{graphicx, txfonts} %txfonts makes the letters sharp
\DeclareGraphicsRule{.tif}{png}{.png}{`convert #1 `dirname #1`/`basename #1 .tif`.png}
\usepackage{epstopdf}
\usepackage{hyperref}
\usepackage{amsthm, amssymb, amsfonts, amsmath, enumerate}
\usepackage{mathrsfs} %Needed for \mathscr
\usepackage{relsize}%, xfrac} % for resizing in math mode


\begin{document}

\title[CS/MATH 335 HW 2]{CS/MATH 335: Probability, Computing, and Graph Theory \\ Homework 5\\ Due in class Monday, October 17, 2016}
\maketitle

{Taylor Heilman and Matt Iammarino}

\textbf{Partner Problems}

Roughly in order of difficulty

\begin{enumerate}

\item \fbox{\parbox{\textwidth}{
Mitzenmacher 1.25}} 
{

}

\item \fbox{\parbox{\textwidth}{
Mitzenmacher 5.7
}}

{





}

\item \fbox{\parbox{\textwidth}{
Mitzenmacher 5.2 

}}

{


}
\item \fbox{\parbox{\textwidth}{
Mitzenmacher 5.3 
}}

{

}
\end{enumerate}

\noindent \textbf{Individual Problems}
\begin{enumerate}

\item[(5)] \fbox{\parbox{\textwidth}{
Mitzenmacher 1.24

}} 

{

\begin{center}
 \begin{tabular}{||c | c | c | c | c||} 
 \hline
   & Edges & Max Deg & Min Degree & Av. Deg  \\ [0.5ex] 
 \hline\hline
 $N_n$ & 0 & 0 & 0 & 0 \\ 
 \hline
 $P_n (n>2)$ & $n$-1 & 2 & 1 & $\frac{2(n-1)}{n}$ \\
 \hline
 $C_n (n>2)$ & $n$ & 2 & 2 & 2 \\
 \hline
 $K_n (n>2)$ & $\frac{n(n-1)}{2}$ & $n-1$ & $n-1$ & $n-1$\\
 \hline
 $K_{n,n} (n>1)$& $n^2$ & $n$ & $n$ & $n$ \\ [1ex] 
 \hline
\end{tabular}
\end{center}

In regards to $P_n, C_n$ and $K_n$, if $n$= 2 then there is only 1 edge and the max, min and average degree is 1. If $n$ = 1 or 0, then there are no edges and the max, min and average degree is 0. For $K_{n,n}$, if $n$=1 then there is only 1 edge and the max, min and average degree is 1. If $n$=0 then there are 0 edges and the max, min and average degree is 0. \newline

(Assuming all graphs have the same number of nodes) A subgraph is made by simply removing edges from the initial graph,  thus it is clear to see that $N_n$ is a subgraph of all the other graphs since removing all edges from any graph will get you $N_n$

$P_n$ is  a subgraph of $C_n, K_n, K_{n,n}$.  Removing any edge from $C_n$ gets you $P_n$.  Since $K_n$ has every edge possible we can obtain $P_n$ by removing the edges to create a path on $n$ nodes.  You can also obtain $P_n$ from $K_{n,n}$  by removing the proper edges.

$C_n$ is a subgraph of $K_n$ and $K_{n,n}$. Since $K_n$ has every edge possible we can obtain $C_n$ by removing the edges to create a cycle on $n$ nodes.  We can also obtain $C_n$ from $K_{n,n}$ by removing the proper edges.

$K_{n,n}$ is a subgraph of $K_n$. Since $K_n$ is maximally connected we can obtain $K_{n,n}$ by removing the proper edges.

Lastly, $K_n$ is only a subgraph of itself since it is maximally connected. \newline

For any of the above graphs, excluding $N_n$, if $n = 2$ then its compliment is $N_n$. This is because there will only be one edge connecting 2 nodes, so the compliment of that graph is $N_2$.

The compliment of $N_n$ is $K_n$ (and vice versa) because $N_n$ has no edges while $K_n$ is maximally connected.  

$C_3$'s compliment is $N_3$
}


\end{enumerate}

\end{document}