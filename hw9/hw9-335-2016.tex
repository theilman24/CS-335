\documentclass[11pt]{amsart}
%\geometry{landscape}                % Activate for for rotated page geometry
\usepackage[parfill]{parskip}    % Activate to begin paragraphs with an empty line rather than an indent
\usepackage{graphicx, txfonts} %txfonts makes the letters sharp
\DeclareGraphicsRule{.tif}{png}{.png}{`convert #1 `dirname #1`/`basename #1 .tif`.png}
\usepackage{epstopdf}
\usepackage{hyperref}
\usepackage{amsthm, amssymb, amsfonts, amsmath, enumerate}
\usepackage{mathrsfs} %Needed for \mathscr
\usepackage{relsize}%, xfrac} % for resizing in math mode



\begin{document}


\title[CS/MATH 335 HW 9]{CS/MATH 335: Probability, Computing, and Graph Theory \\ Homework 9\\ Due in class Thursday, December 1, 2016 \\
One problem (your choice) due Thursday, November 17}
\maketitle


\noindent Show all work to ensure full credit. 
\\

\textbf{Partner Problems}

Roughly in order of difficulty

\begin{enumerate}

\item \fbox{\parbox{\textwidth}{
Mitzenmacher 6.16. 
}}

\item \fbox{\parbox{\textwidth}{
Mitzenmacher 6.11.
}}

\item \fbox{\parbox{\textwidth}{
Section 6.3 analyzed the following algorithm for constructing an independent set: for every $v\in G$, delete $v$ and all edges touching $v$ with probability $1 - 1 /d$. Let $H$ be the set of vertices which survive this process.

\begin{enumerate}
\item Use the method of conditional expectations to turn this algorithm into a deterministic algorithm which always finds an independent set of size $n/2d$.
\item Let G be a 3-regular graph (i.e. all vertices have degree 3). Consider the randomized algorithm that deletes each vertex independently with probability 2/3 as above. For every edge that remains, delete one of its end-points randomly. Derive an upper bound on the probability that this algorithm finds an independent set smaller than $n(1 - \epsilon)/6$. Hint: Chernoff bounds.
\end{enumerate}
}}

\item \fbox{\parbox{\textwidth}{
Mitzenmacher 6.15. Hint: this is the challenge problem. You can do each part independently of the others. So make sure to attempt all parts of this problem.
}} 

BONUS partner problem:

\item \fbox{\parbox{\textwidth}{
Fix a constant $p \in (0,1)$. Prove that almost no graph in $G(n,p)$ has a complete separating subgraph.
}} 

\end{enumerate}

\noindent \textbf{Individual Problems}
\begin{enumerate}

\item[(6)] \fbox{\parbox{\textwidth}{
Mitzenmacher 6.8.
}} 

BONUS individual problem:

\item[(7)] \fbox{\parbox{\textwidth}{
Given $d \in \mathbb{N}$, is there a threshold function for the property of containing a $d$-dimensional cube? If so, what is the threshold function? If not, why not?
}}

\end{enumerate}

\end{document}