\documentclass[11pt]{amsart}
%\geometry{landscape}                % Activate for for rotated page geometry
\usepackage[parfill]{parskip}    % Activate to begin paragraphs with an empty line rather than an indent
\usepackage{graphicx, txfonts} %txfonts makes the letters sharp
\DeclareGraphicsRule{.tif}{png}{.png}{`convert #1 `dirname #1`/`basename #1 .tif`.png}
\usepackage{epstopdf}
\usepackage{hyperref}
\usepackage{amsthm, amssymb, amsfonts, amsmath, enumerate}
\usepackage{mathrsfs} %Needed for \mathscr
\usepackage{relsize}%, xfrac} % for resizing in math mode



\begin{document}


\title[CS/MATH 335 HW 8]{CS/MATH 335: Probability, Computing, and Graph Theory \\ Homework 8\\ Due in class Tuesday, November 15, 2016 \\
Taylor Heilman individual}
\maketitle


\noindent Show all work to ensure full credit. 
\\

\textbf{Partner Problems}

Roughly in order of difficulty. CORRECT VERSION.

\begin{enumerate}

\item \fbox{\parbox{\textwidth}{
Mitzenmacher 6.6.
}}

\item \fbox{\parbox{\textwidth}{
Suppose you are given a graph $G$ and you are 3 coloring the vertices. Prove that there exists a coloring in which two thirds of the edges have distinct colors on their ends.
}}

\item \fbox{\parbox{\textwidth}{
Mitzenmacher 6.7. Hint: sample and modify.
}}

\item \fbox{\parbox{\textwidth}{
Mitzenmacher 6.9. Hint: part (a) is pretty easy. For part (b), follow your nose, looking at a random ranking (there are $n!$) and thinking about how many edges in the tournament disagree with a given ranking. In the middle of your analysis, you may find yourself looking at an event like $P(X_i \geq \frac{51}{100} {n \choose 2})$. This is a job for Chernoff bounds! The ``sufficiently large n'' part comes at the very end, to show that a certain probability is strictly less than 1.
}}

\end{enumerate}

\noindent \textbf{Individual Problems}
\begin{enumerate}

\item[(5)] \fbox{\parbox{\textwidth}{
Streaming Exercises from Ullman's book:
\begin{enumerate}
\item Exercise 4.3.1
\item Exercise 4.5.4
\item Exercise 4.6.1
\item Exercise 4.6.2
\end{enumerate}

}} 

{


a.) Given the fact that we have 8 billion bits, 1 billion members of the set $S$ and 3 hash functions we see that $n= 8(10^9), m = 10^9, k = 3$.  Thus the $pr(bit = 0) = e^{\frac{-3(10^9)}{8(10^9)}}$ which simplifies to $\frac{1}{e^{\frac{3}{8}}}$ and $pr(bit = 1) = (1- e^{\frac{-3(10^9)}{8(10^9)}})$ which simplifies to $(1-\frac{1}{e^{\frac{3}{8}}})$. Hence, the probability of getting a false positive is the probability of hashing to a $1$ from every hash function, which is =  $(1-\frac{1}{e^{\frac{3}{8}}})^3 = .0305$. 

 Following the same thought process the probability of a false positive with 4 hash functions is simply $(1-\frac{1}{e^{\frac{4}{8}}})^4 = .0239$

$\newline$

b.) Given the stream: 3 1 4 1 3 4 2 1 2  

We compute the values of each variable. From these values we use the formula: $n(3v^2-3v+1)$ where $n=9$ and $v=X.val$ to find the estimate.


 we get

$X_1 = X_1.element = 3, X_1.value=2$, estimate = 63

$X_2 = X_2.element = 1, X_2.value=3$, estimate = 171

$X_3 = X_3.element = 4, X_3.value=2$, estimate = 63

$X_4 = X_4.element = 1, X_4.value=2$, estimate = 63

$X_5 = X_5.element = 3, X_5.value=1$, estimate = 9

$X_6 = X_6.element = 4, X_1.value=1$, estimate = 9

$X_7 = X_7.element = 2, X_1.value=2$, estimate = 63

$X_8 = X_8.element = 1, X_1.value=1$, estimate = 9

$X_9 = X_9.element = 2, X_1.value=1$, estimate = 9


the average of the estimate comes out to = 51, which is the exact value of the third moment.  We can confirm this by calculating the third moment: 

$2^3 + 3^3 + 2^3 + 2^3 = 51$



$\newline$



c.) Given the following bit stream: ..1011011000101110110010110 we divide the bitstream into the following buckets:

 $\underline{..101}$ $\underline{101100010}$ $\underline{11101}$ $\underline{1001}$ 0 $\underline{1}$ $\underline{1}$ 0 
 
 To estimate the amount of ones in the last 5 bits we count the first two buckets with size 1, then we count half of the bucket at position $t-4$ that has size=2.  Thus for our estimation get 1+1+1= 3, which is the correct answer of 3.

To estimate the amount of ones in the last 15 bits we follow the same algorithm.  We count the windows at positions $(t-1) size=1,(t-2) size=1,(t-4) size=2, and (t-8) size=4.$ Lastly we count half the window at position $(t-14)$ that has size=4.  Thus we get 1+1+2+4+2= 10 which is also 1 off from the correct answer of 9.

d.) Given the stream $1001011011101$ I found 4 different ways to partition the stream into buckets.  Since there are 8 ones in the stream, we can have..

One bucket of size eight:

$\underline{1 0 0 1 0 1 1 0 1 1 1 0 1}$

Two buckets of size 4:

$\underline{1 0 0 1 0 1 1 }$          0           $\underline{1 1 1 0 1}$

Two buckets of size 2 and one bucket of size 4:

$\underline{1 0 0 1 0 1 1}$         0       $\underline{1 1}$         $\underline{1 0 1}$

Two buckets of size 1, one bucket of size 2, and one bucket of size 4:

$\underline{1 0 0 1 0 1 1}$      0         $\underline{1 1}$        $\underline{1}$      0         $\underline{1}$


}

BONUS Individual Problem:
\item[(6)] \fbox{\parbox{\textwidth}{
Prove that $R(3,3,\dots,3)\leq 3 \cdot r!$, where there are $r$ copies of 3 on the left. THEN, using that result, prove that if $n\geq 3\cdot r!$ then no matter how the set $[n] = \{1,2,\dots,n\}$ is partitioned into $r$ classes, there must be a solution of the equation $x+y=n$ where all three numbers are in the same class. This shows an application of Ramsey Theory to Number Theory.
}} 

\end{enumerate}

\end{document}