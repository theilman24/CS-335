\documentclass[11pt]{amsart}
%\geometry{landscape}                % Activate for for rotated page geometry
\usepackage[parfill]{parskip}    % Activate to begin paragraphs with an empty line rather than an indent
\usepackage{graphicx, txfonts} %txfonts makes the letters sharp
\DeclareGraphicsRule{.tif}{png}{.png}{`convert #1 `dirname #1`/`basename #1 .tif`.png}
\usepackage{epstopdf}
\usepackage{hyperref}
\usepackage{amsthm, amssymb, amsfonts, amsmath, enumerate}
\usepackage{mathrsfs} %Needed for \mathscr
\usepackage{relsize}%, xfrac} % for resizing in math mode


\begin{document}

\title[CS/MATH 335 HW 2]{CS/MATH 335: Probability, Computing, and Graph Theory \\ Homework 1\\ Due in class Friday, September 23, 2016}
\maketitle

{Taylor Heilman and Matt Iamarino}

\textbf{Partner Problems}

Roughly in order of difficulty

\begin{enumerate}

\item \fbox{\parbox{\textwidth}{

}} 

\item \fbox{\parbox{\textwidth}{
Mitzenmacher 2.13
}}
{

a.)  According to the problem we have $2n$ different coupons and $n$ pairs of coupons.  From this information we know that that $X =  \sum_{i=1}^{n} x_i $. We also need to calculate the probability of getting a coupon from a new pair when we have $i-1$ pairs. Notice the the probability of choosing a specific coupon is $ = \frac{1}{2n}$, hence we get $E[x_i] = \frac{1}{2n}(2n - 2(i-1) = \frac{n-i+1}{n}$.    To find  $E[X]$ we do the following:
 $\newline$
 
$\Rightarrow E[X] = E[\sum_{i=1}^{n} x_i ]$

$\newline$
$\Rightarrow  \sum_{i=1}^{n} E[x_i]$

$\newline$
$\Rightarrow  \sum_{i=1}^{n}  \frac{n-i+1}{n}$

$\newline$
$\Rightarrow n \sum_{i=1}^{n}  \frac{1}{n}$

$\newline$

b.) 

}

\item \fbox{\parbox{\textwidth}{
Mitzenmacher 2.7 

}}

{

a.) Using the information provided that $X$ and $Y$ are independent geometric random variables, $Pr(X=Y) = \sum_{x} (1-p)^{x-1}p (1-q)^{x-1}q$
$\newline$

$\Rightarrow \sum_{x} [(1-p)(1-q)]^{x-1}pq$
$\newline$

So, $Pr(X=Y) = \frac{pq}{p+q-pq}$

$\newline$
b.) $E[max(X,Y)] = E[X] + E[Y] - E[min(X,Y)]$. Notice $E[min(X,Y)] = \frac{1}{p+q-pq}$. Hence, $E[max(X,Y)] = \frac{1}{p} + \frac{1}{q} - \frac{1}{p+q-pq}$

$\newline$
c.) To find the $Pr(min(X,Y)=k)$ we can split the event into two events:
$\newline$

$\Rightarrow Pr(X=k, Y \geq k) + Pr(X > k, Y = k)$ 
$\newline$
 
$\Rightarrow Pr(X=k) Pr(Y \geq k) + Pr(X > k) Pr(Y = k)$ 
$\newline$

$\Rightarrow$ Notice $Pr(X>k) = Pr(X \geq k)-Pr(x=k) = (1-p)^{k-1}(1-p)$
$\newline$

 $\Rightarrow Pr(min(X,Y)=k) = (1-p)^{k-1}p(1-q)^{k-1} + (1-p)^{k-1}(1-p)(1-q)^{k-1}q$
$\newline$

$\Rightarrow = [(1-p)(1-q)]^{k-1}(p+q-pq)$

$\newline$
d.) 


 $E[ X |X \leq Y]  = \sum_z x\frac{Pr(X=x) \cap x \leq Y}{Pr(X \leq Y)}$
 
 Now we compute compute $Pr( X \leq Y):$

$\Rightarrow \sum_z Pr(X= z) Pr(z \leq Y)$
$\newline$

$\Rightarrow \sum_z (1-p)^{z-1}p(1-q)^{z-1}$
$\newline$

$\Rightarrow p \sum_z[(1-p-q+pq)]^{z-1}$
$\newline$

$\Rightarrow \frac{p}{p+q-pq}$

$\newline$
Hence we get:
$\newline$

$\Rightarrow \frac{p}{p+q-pq} \sum_x x(1-p)^{x-1}p(1-q)^{x-1}$

$\Rightarrow (p+q-pq) \sum_x x(1-p-q+pq)^{x-1}$

$\Rightarrow  E[ X |X \leq Y] = \frac{1}{p+q -pq}$
}
\item \fbox{\parbox{\textwidth}{
Mitzenmacher 3.7 
}}

{


}
\end{enumerate}

\noindent \textbf{Individual Problems}
\begin{enumerate}

\item[(5)] \fbox{\parbox{\textwidth}{
Mitzenmacher 3.1

}} 

{Since $X$ is being chosen uniformly at random from $[1,n]$ we know that $P(X=x) = \frac{1}{n}$. Using the formula $V(X) = E[X^2] - (E[X])^2$ we get:
$\newline$

$\Rightarrow \sum_{x=1} ^{n} x^2 (\frac{1}{n})$ - $( \sum_{x=1}^{n} x (\frac{1}{n})) ^2$
$\newline$

$\Rightarrow \frac{1}{n} (1+4+9+...+n^2) - ( \frac{1}{n} (1+2+3+...+n) )^2$
$\newline$

$\Rightarrow \frac{1}{n}(\frac{n(n+1)(2n+1)}{6})- \frac{1}{n} ( \frac{n(n+1)}{2})^2$
$\newline$

$\Rightarrow \frac{(n+1)(2n+1)}{6} - \frac{(n+1)^2}{4}$
$\newline$

$\Rightarrow \frac{(n+1)(4n+2-3n-3)}{12}$
$\newline$

$\Rightarrow \frac{(n+1)(n-1)}{12} = \frac{n^2 - 1}{12}$

}

\end{enumerate}

\end{document}